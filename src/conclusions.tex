\section{Conclusions}\label{sec:conclusions}

\pyhf{} is the first pure-Python implementation of the \HiFa{} specification that leverages modern open source $n$-dimensional array libraries as computational backends to exploit automatic differentiation and hardware acceleration to speed up fits and reduce the time to scientific insight.
It provides a Python and command line API for building, inspection, and to perform statistical inference for \HiFa{} models, and its JSON model serialization has enabled publication of full statistical models from the ATLAS collaboration and improved reinterpretations.
As \pyhf{} is an open source library that has been built as part of the Scikit-HEP community project it has been readily adopted by a growing number of other libraries and tools as a computational and inference engine, allowing for improvements in the library API and computational backends to propagate to the broader user community.
Growing community support and interaction, adoption across the broader particle physics community, and rigorous testing from LHC experiments and IRIS-HEP systems has demonstrated that \pyhf{} has become a key component of the growing ecosystem of Pythonic open source scientific tools in particle physics.
